% Цели-задачи-метод-результаты

Целью дипломного проектирования является готовая базы данных и мобильное приложение
для учета заказов пользователя (юридических лиц).

Задачами данного дипломного проектирования является обследование объекта автоматизации (ОА).

Проектирование базы данных (БД) и реализация мобильного приложения ведётся следующими методами:
описываем программу в виде UML диграмы предецентов, диграмы последовательности, диграмы развертывания;
описываем БД в виде логической модели в draw.io; описываем таблицы в виде класcов на языке программирования TypeScript;
реализуем HTTP запросы на фреймворке NestJS; документируем HTTP запросы с помощью Swagger UI;
реализуем мобильное приложение с помощью библиотеки React Native.

Результатом дипломного проектирования является мобильное приложение и готовая база данных,
которая хранит данные о номенклатуре,
хакрактеиристиках номенклатуры,
списка картинок номенклатуры,
данных юридических лиц,
данные о заказе и статусе заказа.

В первом разделе
проводим анализ существующих мобильных приложений
(WildBerries, OZ.by, LC Waikiki, Lamoda, De Facto, Ali Express).
Выбираем базу данных, средства для реализации серверной части (backend) и мобильного приложения (frontend).

Во втором разделе
описываем серверную часть с помощью UML диграммы прецедентов.
Для каждого прецедента расписывается его назначение, исполнители, предусловие, основный и альтернативный HTTP ответ.
Также HTTP запросы показаны на UML диаграмме последовательностей.
Работа базы данных, серверной части и мобильного приложения показана на диаграмме развертывания.
В этом разделе построена логическая модель и приведены таблицы,
в которых перечислены атрибуты и тип данных.

В третьем разделе
представлена документация SwaggerUI, которая играет важную роль в тестировании эндпоинтов.
SwaggerUI предоставляет удобный интерфейс для взаимодействия с эндпоинтами системы,
позволяя разработчикам и тестировщикам отправлять запросы на серверную часть (backend)
и получать соответствующие ответы.
Кроме того, в третьем разделе приведены скриншоты мобильного приложения,
которые позволяют ознакомиться с его внешним видом и пользовательским интерфейсом.
Эти изображения предоставляют визуальное представление о функциональности и возможностях приложения,
позволяя получить наглядное представление о его работе.

В четвёртом разделе
производится сравнение объема функций из каталога с уточненным объемом функций своего проекта.
Осуществляется расчет себестоимости оборудования по различным статьям,
включая отчисления на социальные нужды, материалы и комплектующие,
машинное время, накладные расходы, а также затраты на освоение и сопровождение программного средства.
Также производится расчет плановой прибыли, прогнозируемой цены без налогов, отпускной цены и
расчета прибыли от реализации программного обеспечения за вычетом налога на прибыль.

В приложении А
представлено техническое задание, которое поможет лучше понять требования к проекту.

В приложении Б
представлены данные для проверки базы данных, что позволяет убедиться в корректности ее работы.

В приложении В
можно ознакомиться с результатами данных в БД.

В приложении Г
представлена инструкция по установке необходимого программного обеспечения для успешной настройки и запуска проекта.
Инструкция включает установку NodeJS, установку пакетного менеджера yarn,
а также установку Docker для запуска БД на операционной системе Windows 10.
Эти инструменты необходимы для работы с серверной частью (backend), разработанной на NestTS,
мобильного приложения (frontend), написанного на React Native с использованием TypeScript,
а также веб-сайтом с панелью администратора (frontend), разработанным с использованием NextTS.
Эта инструкция обеспечит удобство и понятность при настройке необходимого программного обеспечения,
что позволит пользователям успешно разрабатывать и демонстрировать проект.

В приложении Д
прикреплен диск, содержащий исходные коды серверной части,
мобильного приложения и веб-сайта с панелью администратора,
а также отчет, оформленный в LaTeX.


\newpage
