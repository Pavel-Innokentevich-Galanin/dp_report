Задачей данного дипломного проекта является
разработка серверной части на базе TypeScript,
разработка мобильного приложения на базе TypeScript
и разработка веб-сайта с панелью администратора на базе TypeScript.

Последовательность расчетов:

\begin{enumerate}
    \item[-] расчёт объёма функций программного модуля;
    \item[-] расчёт полной себестоимости прошраммного продукта.
\end{enumerate}

\subsection{Расчёт объема функций}

Общий объем ПО определяется по формлуле (\ref{equ:v0}) исходя из объёма функций, реализуемой программой.

\begin{equation}
    \label{equ:v0}
    V_0 = \sum^n_{i=0} V_i \text{, где}
\end{equation}

\begin{enumerate}
    \item[-] $V_0$ - общий объем части ПО; 
    \item[-] $V_i$ - объём функций части ПО;  
    \item[-] $n$ - общее число функций.
\end{enumerate}

В том случае, когда на стадии технико-экономического обоснования проекта невозможно рассчитать точный объем функций,
то на данный объём может быть получен на основании прогнозируемой оценки имеющихся фактических данных по анологичным проектам, выполненным ранее,
или применением нормативов по каталогу функций.

По каталогу фукнкий на соновании функций разрабатываемого ПО определяется общий объём ПО.
Также на основе зависимостей от организационных и технологических условий,
был скорректирован объём на основе экспертных оценок.

Уточнённый объём ПО ($V_y$) определяется по формуле (\ref{equ:vi}).

\begin{equation}
    \label{equ:vi}
    V_y = \sum^n_{i=0} V_yi
\end{equation}

\begin{enumerate}
    \item[] $V_yi$ - уточнённый объем отдельной функции в строках исходного кода.
\end{enumerate}

Перечень и объем функций ПО приведём в таблице~\ref{tab:vPO}.

\begin{table}[h!]
    \centering

    \caption{Перечень и объём функций программного обеспечения}
    \label{tab:vPO}

    \begin{tabular}{|p{2cm}|p{6cm}|p{3cm}|p{3cm}|} 
        \hline
        \multicolumn{1}{|c|}{\textbf{№}}
        & \multicolumn{1}{c|}{\textbf{Наименование}}
        & \multicolumn{2}{c|}{\textbf{Объем функции строк}} \\

        \multicolumn{1}{|c|}{\textbf{функции}}
        & \multicolumn{1}{c|}{\textbf{(содержание)}}
        & \multicolumn{2}{c|}{\textbf{исходного кода}} \\ \hline

        \multicolumn{1}{|c|}{\textbf{x}}
        & \multicolumn{1}{c|}{\textbf{x}}
        & \multicolumn{1}{c|}{\textbf{По каталогу $V_i$}}
        & \multicolumn{1}{c|}{\textbf{Уточнённый $V_yi$}} \\ \hline

        \multicolumn{1}{|c|}{101} & Организация ввода информации                            & \multicolumn{1}{r|}{130 } & \multicolumn{1}{r|}{75  } \\ \hline
        \multicolumn{1}{|c|}{102} & Контроль, предварительная обработка и ввод информации   & \multicolumn{1}{r|}{250 } & \multicolumn{1}{r|}{221 } \\ \hline
        \multicolumn{1}{|c|}{207} & Организация поиска и поиск в базе данных                & \multicolumn{1}{r|}{4720} & \multicolumn{1}{r|}{102 } \\ \hline
        \multicolumn{1}{|c|}{209} & Загрузки базы данных                                    & \multicolumn{1}{r|}{2360} & \multicolumn{1}{r|}{100 } \\ \hline
        \multicolumn{1}{|c|}{305} & Формирование файла                                      & \multicolumn{1}{r|}{2130} & \multicolumn{1}{r|}{680 } \\ \hline
        \multicolumn{1}{|c|}{707} & Графический вывод результатов                           & \multicolumn{1}{r|}{420 } & \multicolumn{1}{r|}{214 } \\ \hline
        \multicolumn{1}{|c|}{801} & Простой поиск контента портала                          & \multicolumn{1}{r|}{570 } & \multicolumn{1}{r|}{163 } \\ \hline
        \multicolumn{1}{|c|}{805} & Создание карты сайта                                    & \multicolumn{1}{r|}{76  } & \multicolumn{1}{r|}{8   } \\ \hline
        \multicolumn{1}{|c|}{806} & Сбор статистики и посетителей сайта                     & \multicolumn{1}{r|}{95  } & \multicolumn{1}{r|}{38  } \\ \hline

        \multicolumn{2}{|r|}{$\sum$}                                                        & \multicolumn{1}{r|}{10751} & \multicolumn{1}{r|}{1601} \\ \hline
    \end{tabular}
\end{table}

С учётом информации, указанной в таблице~\ref{tab:vPO}, уточнённый объем ПО составил 1601 строка кода
вместо предпологаемого количества 10751.

Количество строк кода на текущий момент, принимая во внимание, что проекты разрабатывались с нуля \cite{LinuxCloc}:
\begin{enumerate}
    \item[-] серверная часть (backend) - 14056 строк кода;
    \item[-] мобильное приложение (frontend) - 5439 строк кода;
    \item[-] веб-приложение (frontend) - 5120 строк кода;
\end{enumerate}

\subsection{Расчёт полной себестоимости оборудования}

Стоимостная оценка программного средства у разработчика предпологает составление сметы затрат,
которая включает следующие статьи расходом:

\begin{itemize}
    \item отчисления на социальные нужды ($P_{\text{соц}}$);
    \item материалы и комплектующие изделия ($P_{\text{м}}$);
    \item спецоборудование ($P_{\text{с}}$);
    \item машинное время ($P_{\text{мв}}$);
    \item расходы на научные командировки ($P_{\text{нк}}$);
    \item прочие прямые расходы ($P_{\text{пр}}$);
    \item накладные раходы ($P_{\text{нр}}$);
    \item затраты на освоение и сопровождение программного средства ($P_{\text{о}}$ и $P_{\text{со}}$).
\end{itemize}

Полная себестоимость ($C_{\text{п}}$) разработки программного продукта рассчитывается как сумма расходом
по всем статьям с учётом рыночной стоимости аналогиченых продуктов.

Основной статьей расходом на создание программного продукта является заработная плата проекта
(основная и дополнительная) разработчиков (исполнителей)
($\text{ЗП}_{\text{осн}} + \text{ЗП}_{\text{доп}}$),
в число которых принято включать инженеров-программистов,
руководителей проекта, системных архитекторов, дизайнеров, разработчиков баз данных,
Web-мастеров и других специалистов, необходимых для решения специальных задач в команде.

Расчёт заработной платы разработчиков программного продукта начинается с определения:

\begin{itemize}
    \item продолжительности времени разработки ($\text{Ф}_{\text{рв}}$),
    которое устанавливается студентом экспертным путём с учётом сложности,
    новизны программного обеспечения и фактически затраченного времени.
    В данном дипломном прокете $\text{Ф}_{\text{рв}} = 40\text{ дней}$;
    \item количества разработчиков программного обеcпечения.
\end{itemize}

Заработная плата разработчиков определяется как сумма основной и дополнительной заработной платы всех исполнителей.

\subsubsection*{Вычисление основаной заработной платы}

Основная заработная плата каждого исполнителя определеяется по формуле~(\ref{equ:zpOsn}).

\begin{equation}
    \label{equ:zpOsn}
    \text{ЗП}_\text{осн} = T_\text{ст.1р} \cdot \frac{ \text{T}_\text{К} }{ \text{Ф}_\text{эфф.р.в} } \cdot \text{Ф}_\text{рв} \cdot \text{К}_\text{пр} \text{, где}
\end{equation}

\begin{enumerate}
    \item[-] $T_\text{ст.1р}$ - месячная тарифная ставка, Br
    ($T_\text{ст.1р} = \text{Br } 45.00$);
    \item[-] $\text{T}_\text{К}$ - тарифный коэффициент согласно разряду исполнителя \cite{RBMesTarifStavka2019} \cite{RBMesTarifStavka2023}
    ($\text{T}_\text{К} = 1.35$ - для третьего разряда);
    \item[-] $\text{Ф}_\text{эфф.р.в}$ - фонд рабочего времени исполнителя (продолжительность разработки программного модуля), дни
    ($\text{Ф}_\text{эфф.р.в} = 5\text{ дней}$);
    \item[-] $\text{К}_\text{пр}$ - коэффициент премии, $\text{К}_\text{пр} = 1.5$.
\end{enumerate}

Вычисляем: $\text{ЗП}_\text{осн} = \text{Br } 45.00 \cdot \frac{ 1.35 }{ 7 \text{ дней}  } \cdot 20 \text{ дней} \cdot 1.2 \approx \text{Br } 416.57$.

\subsubsection*{Вычисление часовой тарифной ставки}

Часовая тарифная ставка определеяется по формуле (\ref{equ:tChasStav}).

\begin{equation}
    \label{equ:tChasStav}
    \text{Т}_\text{час.ст.1р} = \frac{ \text{T}_\text{мес.ст.1р.} \cdot 12 }{ \text{П}_\text{рв.} } \text{, где}
\end{equation}

\begin{enumerate}
    \item[-] $T_\text{мес.ст.1р}$ - месячная тарифная ставка рабочего 1 разряда (на 1 января 2023 года - \textbf{Br 223.00} \cite{RBstavka});
    \item[-] $\text{П}_\text{рв.}$ - расчётная норма рабоченго времени (в часах) за год,
    определеяется по производственному календарю текущего года
    Норма рабочего времени при пятидневной рабочей неделе в 2023 г. составит \cite{RBnormRabVrem}:
    \begin{enumerate}
        \item \textbf{при 40-часовой рабочей неделе – 2011 часов} \\
        ($8 \text{ часов} \cdot 252 \text{ дня} - 5 \text{ часов} = 2011 \text{ часов}$);
        \item при 36-часовой рабочей неделе – 1809,4 часа \\
        ($7.2 \text{ часа} \cdot 252 \text{ дня} - 5 \text{ часов} = 1809.4 \text{ часа}$);
        \item при 35-часовой рабочей неделе – 1759 часов \\
        ($7 \text{ часов} \cdot 252 \text{ дня} - 5 \text{ часов} = 1759 \text{ часа}$).
    \end{enumerate}
\end{enumerate}

Вычисляем: $\text{Т}_\text{час.ст.1р} = \frac{ 223.00 \cdot 12 }{ 2011 } \approx 1.33 \text{ (Br/час)}$

\subsubsection*{Вычисление дополнительной заработной платы}

Дополнительная заработная плата каждого исполнителя
($\text{Н}_\text{доп.зп} = 20\%$).
Она рассчитывается от основной заработной платы по формуле (\ref{equ:ZPdop}).

\begin{equation}
    \label{equ:ZPdop}
    \text{ЗП}_\text{доп} = \text{ЗП}_\text{осн} \cdot \frac{ \text{Н}_\text{доп.зп.} }{ 100\% }
\end{equation}

Вычисляем: $\text{ЗП}_\text{доп} = \text{Br }416.57 \cdot \frac{ 20\% }{ 100\% } \approx \text{Br } 83.31$.

Результаты занесём в таблицу \ref{tab:ZPRezult}.

\begin{table}[ht]
    \centering

    \caption{Расчёт заработной платы}
    \label{tab:ZPRezult}

    \begin{tabular}{|p{3cm}|p{1.7cm}|p{1cm}|p{2cm}|p{1cm}|p{1.5cm}|p{1.5cm}|p{1.5cm}|}
        \hline
        \textbf{Категория работников}
        & \textbf{Разряд}
        & \textbf{$\text{Т}_\text{к}$}
        & \textbf{$\text{Ф}_\text{эфф.р.в}$} \textbf{(дн.)}
        & \textbf{$\text{К}_\text{пр}$}
        & \textbf{$\text{ЗП}_\text{осн}$} \textbf{(Br)}
        & \textbf{$\text{ЗП}_\text{доп}$} \textbf{(Br)}
        & \textbf{Всего} \textbf{(Br)}
        \\ \hline

        Инженер-программист
        & 3
        & 1.35
        & 44
        & 1.2
        & 416.57
        & 83.31
        & 499.88
        \\ \hline

        \multicolumn{5}{|r|}{\textbf{Всего}}
        & \textbf{416.57}
        & \textbf{83.31}
        & \textbf{499.88}
        \\ \hline
    \end{tabular}
\end{table}

Таким образом, как видно из таблицы~\ref{tab:ZPRezult}, заработная плата инженера-программиста составляет Br 499.88.

\subsubsection*{Вычисления отчисления на социальные нужды}

Отчисления на социальные нужды ($P_\text{соц}$) определяется по формуле (\ref{equ:Psoc}).

\begin{equation}
    \label{equ:Psoc}
    P_\text{соц} = (\text{ЗП}_\text{осн} + \text{ЗП}_\text{доп}) \cdot \frac{ 34.6\% }{ 100\% }
\end{equation}

Вычисляем: $P_\text{соц} = (\text{Br }416.57 + \text{Br }83.31) \cdot \frac{ 34.6\% }{ 100\% } \approx \text{Br }172.96$.

\subsubsection*{Вычисления расходов на спецоборудование}

Расходы по статье <<Спецоборудование>> ($P_\text{с}$) включает затраты на приобретение технических и программных средств специального назначения,
необходимых для разработчки методического пособия, включая расходы на проектирование, изготовление, откладку и другое.

В данном дипломном проекте для разработки ПО приобретение какого-либо спецоборудования не предусматривалось.
Так как спецоборудование не было приобретено, данная статья не рассчитывается.

$P_\text{с} = \text{Br }0.00$.

\subsubsection*{Вычисления расходов на материалы и комплектующие изделия}

По статье <<Материалы и комплектующие изделия>> ($P_\text{м}$) отражаются расходы на магнитные носители, бумагу, красящие ленты и другие материалы,
необходимые для разработки ПО.
Норма расхода материалов в суммарном выражении определяется в расчете на 100 строк исходного кода по формуле (\ref{equ:Rm}).

\begin{equation}
    \label{equ:Rm}
    P_\text{м} = H_\text{м} \cdot \frac{ V_y }{ 100 } \text{, где}
\end{equation}

\begin{enumerate}
    \item[-] $V_y$ - уточнённый общий объем функции строк исходного кода.
    Согласно расчетам формулы (\ref{equ:vi}) данное значение равно 698 строк;
    \item[-] $H_\text{м}$ - норма расхода материалов в расчёте на 100 строк исходного кода, принимается равной Br 1.00.
\end{enumerate}

Вычисляем: $P_\text{м} = \text{Br } 1.00 \cdot \frac{ 1601 }{ 100 } = 16.01$.

\subsubsection*{Оплата машинного времени}

По статье <<Машинное время>> ($P_\text{мв}$) включают оплату машинного времени, необходимого для разработки и отладки ПО.
Они определяются в машино-часах по нормативам на 100 строк исходного кода машинного времени.
$P_\text{мв}$ определяется по формуле (\ref{equ:Pmv}).

\begin{equation}
    \label{equ:Pmv}
    P_\text{мв} = \text{Ц}_\text{мвi} \cdot \frac{ V_y }{ 100 } \cdot H_\text{мв} \text{, где}
\end{equation}

\begin{enumerate}
    \item[-] $\text{Ц}_\text{мвi}$ - цена одного машинного часа (Br 1.60); 
    \item[-] $V_y$ - уточнённый общий объём машинного кода;
    \item[-] $H_\text{мв}$ - норматив расхода машинного времени на отладку 100 строк кода, машино-часов. Принимается в размере 0.8.
\end{enumerate}

Вычисляем: $P_\text{мв} = \text{Br } 1.60 \cdot \frac{ 1601 }{ 100 } \cdot 0.8 \approx \text{Br }20.49$

\subsubsection*{Научные командировки}

Расходы по статье <<Научные командировки>> ($P_\text{нк}$) берутся либо по смете научных командировок,
разрабатываемой на предприятии, либо в процентах от основной заработной платы исполнителей (10-15\%).

Так как в данном проекте научные командировки не предусмотрены, данная статься не рассчитывается. 

$P_\text{нк} = \text{Br }0.00$.

\subsubsection*{Прочие затраты}

Расходы по статье <<Прочие затраты>> ($P_\text{пр}$) включают затраты на приобретение специальной научно-технической информации и специальной литературы.
Определяются по нормативу в процентах к основной заработной плате исполнителей.

Так как специальная научно-техническая информация и специальная литература не приобреталась, то данная статься не рассчитывается.

$P_\text{пз} = \text{Br }0.00$.

\subsubsection*{Накладные расходы}

Затраты по статье <<Накладные расходы>> ($P_\text{нр}$) связаны с содержанием вспомогательных хозяйств,
а также с расходами на общехозяйственные нужды.
Определяется по нормативу в процентах к основной заработной плате по формуле~(\ref{equ:Pnr}).

\begin{equation}
    \label{equ:Pnr}
    P_\text{нр} = \frac{ H_\text{нр} }{ 100\% } \cdot \text{ЗП}_\text{осн} \text{, где}
\end{equation}

$H_\text{нр}$ - норматив накладных расходов, \%. В данном дипломном проекте норматив накладных расходов равен 40\%.

Вычисляем: $P_\text{нр} = \frac{ 40\% }{ 100\% } \cdot \text{Br }416.57 \approx \text{Br }116.63$.

\subsubsection*{Сумма расходов}

Сумма расходов на программный продукт служит исходной базой ждя расчёта затрат на освоение и сопровождение программного продукта.
Они рачситываются по формуле (\ref{equ:sz}).

\begin{equation}
    \label{equ:sz}
    \text{СЗ} = \text{ЗП}_\text{осн} + \text{ЗП}_\text{доп} + P_\text{соц} + P_\text{с} + P_\text{м} + P_\text{мв}
    + P_\text{нк} + P_\text{пр} + P_\text{нр}
\end{equation}

Вычисляем: $\text{СЗ} = \text{Br }416.57 + \text{Br }83.31 + \text{Br }172.96 + \text{Br }0.00 + \text{Br }16.01 \text{ }+$
$+ \text{Br }20.49 + \text{Br }0.00 + \text{Br }0.00 + \text{Br } 166.63 = \text{Br }875.97$.

\subsubsection*{Затраты на освоение программного продукта}

Затраты на освоение программного продукта ($P_o$).
Организация-разработ-чик участвует в освоении программного продукта и несёт соответствующие затраты,
на которые составляется смета, оплачиваемая заказчиком по договору.
Затраты на освоение определяются по установленному нормативу от суммы затрат по формуле (\ref{equ:Po}). 

\begin{equation}
    \label{equ:Po}
    P_o = \text{СЗ} \cdot \frac{ H_o }{ 100\% } \text{, где}
\end{equation}

\begin{enumerate}
    \item[-] $\text{СЗ}$ - сумма расходов по статьям на разработку ПО, Br; 
    \item[-] $H_o$ - установленный норматив затрат на освоение, \%. Для данного дипломного проекта принимается равной 5\%.
\end{enumerate}

Вычисляем: $P_o = \text{Br }875.97 \cdot \frac{ 5\% }{ 100\% } \approx \text{Br } 43.80$.

\subsubsection*{Затраты на сопровождение программного продукта}

Затраты на сопровождение программного продукта ($P_{co}$).
Организация-разработчик осуществляет сопровождение программного продукта и несёт расходы,
которые оплачиваются заказчиком в соответствии с договором и сметой на сопровождение.
Эти расходы рассчитываются по формуле (\ref{equ:Pco}).

\begin{equation}
    \label{equ:Pco}
    P_{co} = \text{СЗ} \cdot \frac{ H_{co} }{ 100\% } \text{, где}
\end{equation}

\begin{enumerate}
    \item[-] $\text{СЗ}$ - сумма расходов по статьям на разработку ПО, Br; 
    \item[-] $H_{co}$ - установленный норматив затрат на сопровождение ПО, \%. Для данного дипломного проекта принимаемой равной 5\%. 
\end{enumerate}

Вычисляем: $P_{co} = \text{Br }875.97 \cdot \frac{ 5\% }{ 100\% } \approx \text{Br } 43.80$.

\subsubsection*{Полная себестоимость}

Полная себестоимость ($\text{СП}$) разработки ПО рассчитывается как сумма расходов по всем статьям.
Она определяется по формуле~(\ref{equ:SP}). 

\begin{equation}
    \label{equ:SP}
    \text{СП} = \text{СЗ} + P_o + P_{co}
\end{equation}

Вычисляем: $\text{СП} = \text{Br }875.57 + \text{Br }43.80 + \text{Br }43.80 = \text{Br }963.17$.

В результате всех расчётов полная себестоимость ПО составила $\text{Br }963.17$.

\subsection{Расчёт цены и прибыли}

\subsubsection*{Рассчёт плановой прибыли}

Для определения цены программного продукта необходимо расчитать плановую прибыль, которая рассчитывается по формуле (\ref{equ:P}).

\begin{equation}
    \label{equ:P}
    \text{П} = \text{СП} \cdot \frac{ R }{ 100\% } \text{, где}
\end{equation}

$R$ - уровень рентабельности ПО, \%. В данном дипломном проекте уровень рентабельности равен 20\%.

Вычисляем: $\text{П} = \text{Br }963.17 \cdot \frac{ 20\% }{ 100\% } \approx \text{Br }192.63$.

\subsubsection*{Рассчёт прогнозируемой цены продукта без налогов}

После расчёта прибыли от реализации по формуле (\ref{equ:Cep}) определяется прогнозируемая цена программного продукта без налогов.

\begin{equation}
    \label{equ:Cep}
    \text{Ц}_\text{п} = \text{СП} + \text{П} \text{, где}
\end{equation}

\begin{enumerate}
    \item[-] $\text{СП}$ - полная себестоимость ПО на базе TypeScript, Br;
    \item[-] $\text{П}$ - плановая прибыль от реализации ПО, Br. 
\end{enumerate}

Вычисляем: $\text{Ц}_\text{п} = \text{Br }963.17 + \text{Br }192.63 = \text{Br }1155.80$.

\subsubsection*{Рассчёт отпускной цены}

Отпускная цена (цена реализации) программного продукта включает налог на добавленную стоимость и рассчитывается по формуле~(\ref{equ:Ceo}).

\begin{equation}
    \label{equ:Ceo}
    \text{Ц}_\text{о} = \text{СП} + \text{П} + \text{НДС}_\text{пп} \text{, где}
\end{equation}

\begin{enumerate}
    \item[-] $\text{СП}$ - полная себестоимость ПО на базе TypeScript, Br;
    \item[-] $\text{П}$ - плановая прибыль от реализации ПО, Br;
    \item[-] $\text{НДС}_\text{пп}$ - налог на добавленную стоимость, Br.
\end{enumerate}

Для данного программного продукта $\text{НДС}_\text{пп}$ рассчитывается по формуле~(\ref{equ:NDSpp}).

\begin{equation}
    \label{equ:NDSpp}
    \text{НДС}_\text{пп} = \text{Ц}_\text{п} \cdot \frac{ \text{НДС} }{ 100\% } \text{, где}
\end{equation}

\begin{enumerate}
    \item[-] $\text{Ц}_\text{п}$ - прогнозируемая цена, Br;
    \item[-] $\text{НДС}$ - налог на добавленную стоимость, \%. В настоящее время 20\%.
\end{enumerate}

Вычисляем:
\begin{enumerate}
    \item[-] $\text{НДС}_\text{пп} = \text{Br }1155.80 \cdot \frac{ 20\% }{ 100\% } = \text{Br } 231.16$;
    \item[-] $\text{Ц}_\text{о} = \text{Br }963.17 + \text{Br }192.63 + \text{Br }231.16 = \text{Br }1386.96$.
\end{enumerate}

\subsubsection*{Рассчёт прибыли от реализации ПО за вычетов налога на прибыль}

Прибыль от реализации программного продукта за вычетом налога на прибыль является чистой прибылью ($\text{ПЧ}$).
Чистая прибыль остаётся организации-разработчику и представляет собой экономический эффект от создания нового программного продукта.
Она рассчитывается по формуле (\ref{equ:PCh}).

\begin{equation}
    \label{equ:PCh}
    \text{ПЧ} = \text{П} \cdot ( 1 - \frac{ \text{Н}_\text{п} }{ 100\% }) \text{, где}
\end{equation}

\begin{enumerate}
    \item[-] $\text{П}$ - плановая прибыль от реализации программного модуля, Br;
    \item[-] $\text{Н}_\text{п}$ - ставка налога на прибыль, \%. В настоящее время он равен 18\%.
\end{enumerate}

Вычисляем: $\text{ПЧ} = \text{Br }192.63 \cdot ( 1 - \frac{ 18\% }{ 100\% }) \approx \text{Br }157.96$.

Расчеты, связанные с ценой и прибылью ПО, представлены в таблице~\ref{tab:RaschetOtpusknoiCeniIChistoiPribiliPO}.

\begin{table}[ht]
    \centering

    \caption{Расчёт отпускной цены и чистой прибыли}
    \label{tab:RaschetOtpusknoiCeniIChistoiPribiliPO}

    \begin{tabular}{|l|r|r|}
        \hline
        \multicolumn{1}{|c|}{ \textbf{Наименование статей затрат} }
        & \multicolumn{1}{c|}{ \textbf{Норматив, \%} }
        & \multicolumn{1}{c|}{ \textbf{Сумма затрат, Br} }
        \\ \hline

        Полная себестоимость ($\text{СП}$) & - & 963.17 \\ \hline
        Прибыль ($\text{П}$) & 20 & 192.63 \\ \hline
        Цена без НДС ($\text{Ц}_\text{п}$) & - & 1155.80 \\ \hline
        НДС ($\text{НДС}_\text{пп}$) & 20 & 231.16\\ \hline
        Отпускная цена, ($\text{Ц}_\text{о})$ & - & 1386.96 \\ \hline
        Налог на прибыль ($\frac{ \text{П} \cdot H_\text{п} }{ 100 }$) & 18 & 34.67 \\ \hline
        Чистая прибыль ($\text{ПЧ})$ & - & 157.96 \\ \hline
    \end{tabular}
\end{table}

В ходе произведенных расчетов определены основные экономические показатели:
\begin{enumerate}
    \item[-] полная себестоимость - Br 963.17;
    \item[-] прогнозируемая цена - Br 1155.80;
    \item[-] Чистая прибыль - Br 157.96.
\end{enumerate}

При расчете цены учтены отчисления в фонд социальной защиты, а также налоги, необходимые к уплате.
К конечному итогу получаем окончательную цену продукта, равную Br 1386.96. 
