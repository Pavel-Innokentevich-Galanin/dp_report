Задачей данного дипломного проекта является
разработка серверной части на NestJS на языке программирования TypeScript,
разработка мобильного приложения на React Native на языке программирования TypeScript
и разработка веб-приложения с панелью администратора и менеджера на Create-React-App на языке программирования TypeScript.

Последовательность расчетов:

\begin{enumerate}
    \item[-] расчёт объёма функций программного модуля;
    \item[-] расчёт полной себестоимости прошраммного продукта.
\end{enumerate}

\subsection{Расчёт объема функций}

Общий объем ПО определяется по формлуле (\ref{equ:v0}) исходя из объёма функций, реализуемой программой.

\begin{equation}
    \label{equ:v0}
    \text{V}_0 = \sum^\text{n}_{\text{i}=0} \text{V}_\text{i} \text{, где}
\end{equation}

\begin{enumerate}
    \item[-] $\text{V}_0$ - общий объем части ПО; 
    \item[-] $\text{V}_\text{i}$ - объём функций части ПО;  
    \item[-] $\text{n}$ - общее число функций.
\end{enumerate}

В том случае, когда на стадии технико-экономического обоснования проекта невозможно рассчитать точный объем функций,
то на данный объём может быть получен на основании прогнозируемой оценки имеющихся фактических данных по анологичным проектам, выполненным ранее,
или применением нормативов по каталогу функций.

По каталогу фукнкий на соновании функций разрабатываемого ПО определяется общий объём ПО.
Также на основе зависимостей от организационных и технологических условий,
был скорректирован объём на основе экспертных оценок.

Уточнённый объём ПО ($\text{V}_\text{y}$) определяется по формуле (\ref{equ:vi}).

\begin{equation}
    \label{equ:vi}
    \text{V}_\text{y} = \sum^\text{n}_{\text{i}=0} \text{V}_\text{yi}
\end{equation}

\begin{enumerate}
    \item[] $\text{V}_\text{yi}$ - уточнённый объем отдельной функции в строках исходного кода.
\end{enumerate}

Перечень и объем функций ПО приведём в таблице~\ref{tab:vPO}.

\begin{table}[h!]
    \centering\small

    \caption{Перечень и объём функций программного обеспечения}
    \label{tab:vPO}

    \begin{tabular}{|p{1cm}|p{8.5cm}|p{3cm}|p{3cm}|} 
        \hline
        \multicolumn{1}{|c|}{№}
        & \multicolumn{1}{c|}{Наименование}
        & \multicolumn{2}{c|}{Объем функции строк} \\

        \multicolumn{1}{|c|}{функции}
        & \multicolumn{1}{c|}{(содержание)}
        & \multicolumn{2}{c|}{исходного кода} \\ \hline

        \multicolumn{1}{|c|}{x}
        & \multicolumn{1}{c|}{x}
        & \multicolumn{1}{c|}{По каталогу $\text{V}_\text{i}$}
        & \multicolumn{1}{c|}{Уточнённый $\text{V}_\text{yi}$} \\ \hline

        \multicolumn{1}{|c|}{101} & Организация ввода информации                            & \multicolumn{1}{r|}{130 } & \multicolumn{1}{r|}{75  } \\ \hline
        \multicolumn{1}{|c|}{102} & Контроль, предварительная обработка и ввод информации   & \multicolumn{1}{r|}{250 } & \multicolumn{1}{r|}{221 } \\ \hline
        \multicolumn{1}{|c|}{207} & Организация поиска и поиск в базе данных                & \multicolumn{1}{r|}{4720} & \multicolumn{1}{r|}{102 } \\ \hline
        \multicolumn{1}{|c|}{209} & Загрузки базы данных                                    & \multicolumn{1}{r|}{2360} & \multicolumn{1}{r|}{100 } \\ \hline
        \multicolumn{1}{|c|}{305} & Формирование файла                                      & \multicolumn{1}{r|}{2130} & \multicolumn{1}{r|}{680 } \\ \hline
        \multicolumn{1}{|c|}{707} & Графический вывод результатов                           & \multicolumn{1}{r|}{420 } & \multicolumn{1}{r|}{214 } \\ \hline
        \multicolumn{1}{|c|}{801} & Простой поиск контента портала                          & \multicolumn{1}{r|}{570 } & \multicolumn{1}{r|}{163 } \\ \hline
        \multicolumn{1}{|c|}{805} & Создание карты сайта                                    & \multicolumn{1}{r|}{76  } & \multicolumn{1}{r|}{8   } \\ \hline
        \multicolumn{1}{|c|}{806} & Сбор статистики и посетителей сайта                     & \multicolumn{1}{r|}{95  } & \multicolumn{1}{r|}{38  } \\ \hline

        \multicolumn{1}{|c|}{-}   & Всего                                                   & \multicolumn{1}{r|}{10751}& \multicolumn{1}{r|}{1601} \\ \hline
    \end{tabular}
\end{table}

С учётом информации, указанной в таблице~\ref{tab:vPO}, уточнённый объем ПО составил 1601 строка кода
вместо предпологаемого количества 10751.

Количество строк кода на текущий момент, принимая во внимание, что проекты разрабатывались с нуля \cite{LinuxCloc}:
\begin{enumerate}
    \item[-] серверная часть (backend) - 16340 строк кода;
    \item[-] мобильное приложение (frontend) - 5748 строк кода;
    \item[-] веб-приложение с панелью администратора (frontend) - 4529 строк кода;
    \item[-] веб-приложение с панелью менеджера (frontend) - 1897 строк кода.
\end{enumerate}

\subsection{Расчёт полной себестоимости}

Стоимостная оценка программного средства у разработчика предпологает составление сметы затрат,
которая включает следующие статьи расходов:

\begin{itemize}
    \item заработную плату исполнителей (основную - $\text{ЗП}_{\text{осн}}$ и дополнительную - $\text{ЗП}_{\text{доп}}$);
    \item отчисления на социальные нужды ($\text{P}_{\text{соц}}$);
    \item материалы и комплектующие изделия ($\text{P}_{\text{м}}$);
    \item спецоборудование ($\text{P}_{\text{с}}$);
    \item машинное время ($\text{P}_{\text{мв}}$);
    \item расходы на научные командировки ($\text{P}_{\text{нк}}$);
    \item прочие прямые расходы ($\text{P}_{\text{пр}}$);
    \item накладные раходы ($\text{P}_{\text{нр}}$);
    \item затраты на освоение и сопровождение программного средства ($\text{P}_{\text{о}}$ и $\text{P}_{\text{со}}$).
\end{itemize}

Полная себестоимость ($\text{С}_{\text{п}}$) разработки программного продукта рассчитывается как сумма расходов
по всем статьям с учётом рыночной стоимости аналогиченых продуктов.

Основной статьей расходов на создание программного продукта (ПП) является заработная плата проекта
(основная и дополнительная) разработчиков (исполнителей)
($\text{ЗП}_{\text{осн}} + \text{ЗП}_{\text{доп}}$),
в число которых принято включать инженеров-программистов,
руководителей проекта, системных архитекторов, дизайнеров, разработчиков баз данных,
Web-мастеров и других специалистов, необходимых для решения специальных задач в команде.

Расчёт заработной платы разработчиков программного продукта (ПП) начинается с определения:

\begin{itemize}
    \item продолжительности времени разработки ($\text{Ф}_{\text{рв}}$),
    которое устанавливается студентом экспертным путём с учётом сложности,
    новизны программного продукта (ПП) и фактически затраченного времени.
    В данном дипломном прокете $\text{Ф}_{\text{рв}} = 53\text{ дня}$;
    \item количества разработчиков программного продукта (ПП), которое может варьироваться от 1 до 4 человек.
\end{itemize}

Заработная плата разработчиков определяется как сумма основной и дополнительной заработной платы всех исполнителей.

\subsubsection*{Вычисление основаной заработной платы}

Основная заработная плата каждого исполнителя определяется по формуле~(\ref{equ:zpOsn}).

\begin{equation}
    \label{equ:zpOsn}
    \text{ЗП}_\text{осн} = \text{Т}_\text{ст1р} \cdot \frac{ \text{Т}_\text{к} }{ 22 } \cdot \text{Ф}_\text{рв} \cdot \text{К}_\text{пр} \text{, где}
\end{equation}

\begin{enumerate}
    \item[-] $\text{Т}_\text{ст1р}$ - месячная тарифная ставка 1 разряда рабочего,
    утвержденная согласно ЕТС РБ на дату написания дипломного проекта;
    \item[-] $\text{Т}_\text{к}$ - тарифный коэффициент согласно разряду исполнителя;
    \item[-] $\text{Ф}_\text{рв}$ - фонд рабочего времени исполнителя (продолжительность разработки программного продукта, дни);
    \item[-] $\text{К}_\text{пр}$ - коэффициент премии, где $\text{К}_\text{пр} \in [1.2; 1.5]$.
\end{enumerate}

Пусть $\text{Т}_\text{ст1р} = \text{Br } 228.00$ как месячная тарифная ставка на 1 января 2023 года \cite{economic_BazovayaStavkaMyFin} \cite{economic_BazovayaStavka} \cite{economic_BazovayaStavkaPravoBy} \cite{economic_BazovayaStavkaPravoByPdf}.

Пусть $\text{Т}_\text{к} = 2.48$ как 10-ый разряд специалиста \cite{economic_TarifniKoeficientPravoBy} \cite{economic_TarifniKoeficientPravoByPdf}.

Пусть $\text{Ф}_\text{рв} = 53 \text{ дня}$ как продолжительность разработки с 23 марта 2023 по 13 мая 2023 года.

Пусть $\text{К}_\text{пр} = 1.2$ - коэффициент премии, где $\text{К}_\text{пр} \in [1.2; 1.5]$.

Вычисляем: $\text{ЗП}_\text{осн} = \text{Br } 228.00 \cdot 2.48 \cdot \frac{ 1 }{ 22 \text{ дня} } \cdot 53 \text{ дня} \cdot 1.2 \approx \text{Br } 1634.64$.

\subsubsection*{Вычисление часовой тарифной ставки}

Часовая тарифная ставка определеяется по формуле (\ref{equ:tChasStav}).

\begin{equation}
    \label{equ:tChasStav}
    \text{Т}_\text{час.ст.1р} = \frac{ \text{T}_\text{мес.ст.1р.} \cdot 12 }{ \text{П}_\text{рв.} } \text{, где}
\end{equation}

\begin{enumerate}
    \item[-] $\text{Т}_\text{ст1р}$ - месячная тарифная ставка;
    \item[-] $\text{П}_\text{рв}$ - расчётная норма рабоченго времени (в часах) за год,
    определеяется по производственному календарю текущего года.
\end{enumerate}

Норма рабочего времени при пятидневной рабочей неделе в 2023 году составит \cite{RBnormRabVrem}:
\begin{enumerate}
    \item при 40-часовой рабочей неделе – 2011 часов \\
    ($8 \text{ часов} \cdot 252 \text{ дня} - 5 \text{ часов} = 2011 \text{ часов}$);
    \item при 36-часовой рабочей неделе – 1809,4 часа \\
    ($7.2 \text{ часа} \cdot 252 \text{ дня} - 5 \text{ часов} = 1809.4 \text{ часа}$);
    \item при 35-часовой рабочей неделе – 1759 часов \\
    ($7 \text{ часов} \cdot 252 \text{ дня} - 5 \text{ часов} = 1759 \text{ часа}$).
\end{enumerate}

Вычисляем: $\text{Т}_\text{час.ст1р} = \frac{ 228.00 \cdot 12 }{ 2011 } \approx 1.36 \text{ (Br/час)}$

\subsubsection*{Вычисление дополнительной заработной платы}

Дополнительная заработная плата каждого исполнителя
рассчитывается от основной заработной платы по формуле (\ref{equ:ZPdop}).

\begin{equation}
    \label{equ:ZPdop}
    \text{ЗП}_\text{доп} = \text{ЗП}_\text{осн} \cdot \frac{ \text{Н}_\text{доп.зп.} }{ 100\% }
\end{equation}

где $\text{Н}_\text{доп.зп} \in [10\%; 20\%]$.

Пусть $\text{Н}_\text{доп.зп} = 10\%$.

Вычисляем: $\text{ЗП}_\text{доп} = \text{Br }1634.64 \cdot \frac{ 10\% }{ 100\% } \approx \text{Br } 163.46$.

\subsubsection*{Результаты вычисления заработной платы}

Результаты вычисления заработной платы отображены в таблице~\ref{tab:ZPRezult}.

\begin{table}[ht]
    \centering

    \caption{Расчёт заработной платы}
    \label{tab:ZPRezult}

    \begin{tabular}{|p{4.8cm}|p{1.5cm}|p{1cm}|p{1.5cm}|p{1cm}|p{1.5cm}|p{1.5cm}|p{1.5cm}|}
        \hline
        \multicolumn{1}{|c|}{Категория}
        & \multicolumn{1}{c|}{Разряд}
        & \multicolumn{1}{c|}{$\text{Т}_\text{к}$}
        & \multicolumn{1}{c|}{$\text{Ф}_\text{р.в}$}
        & \multicolumn{1}{c|}{$\text{К}_\text{пр}$}
        & \multicolumn{1}{c|}{$\text{ЗП}_\text{осн}$}
        & \multicolumn{1}{c|}{$\text{ЗП}_\text{доп}$}
        & \multicolumn{1}{c|}{Всего}
        \\
        \multicolumn{1}{|c|}{работников}
        & 
        & 
        & \multicolumn{1}{c|}{(дни)}
        & 
        & \multicolumn{1}{c|}{(Br)}
        & \multicolumn{1}{c|}{(Br)}
        & \multicolumn{1}{c|}{(Br)}
        \\ \hline

        Инженер-программист
        & 10
        & 2.38
        & 53
        & 1.2
        & \multicolumn{1}{r|}{1634.64}
        & \multicolumn{1}{r|}{163.46}
        & \multicolumn{1}{r|}{1798.10}
        \\ \hline

        Всего
        & -
        & -
        & -
        & -
        & \multicolumn{1}{r|}{1634.64}
        & \multicolumn{1}{r|}{163.46}
        & \multicolumn{1}{r|}{1798.10}
        \\ \hline
    \end{tabular}
\end{table}

Таким образом, как видно из таблицы~\ref{tab:ZPRezult}, заработная плата инженера-программиста составляет Br 1798.10.

\subsubsection*{Вычисление отчисления на социальные нужды}

Отчисления на социальные нужды ($\text{P}_\text{соц}$) определяется по формуле (\ref{equ:Psoc})
в соотвествии с действующим законодательством по нормативу
(29\% - отчисления в ФСЗН + 6\% отчисления по обязательному страхованию).

\begin{equation}
    \label{equ:Psoc}
    \text{Р}_\text{соц} = (\text{ЗП}_\text{осн} + \text{ЗП}_\text{доп}) \cdot \frac{ 29\% + 6\% }{ 100\% }
\end{equation}

Вычисляем: $\text{Р}_\text{соц} = (\text{Br } 1634.64 + \text{Br } 163.46) \cdot \frac{ 29\% + 6\% }{ 100\% } \approx \text{Br } 629.34$.

\subsubsection*{Вычисление расходов на спецоборудование}

Расходы по статье <<Спецоборудование>> ($\text{Р}_\text{с}$) включает затраты на приобретение технических и программных средств специального назначения,
необходимых для разработки конкретного программного продукта (ПП),
включая расходы на проектирование, изготовление, откладку и другое.
Определяется по фактически действующим на рынке ценам.
В тех случаях, когда спецоборудование не приобретается, данныя статья не расчитывается.

В данном дипломном проекте для разработки ПО приобретение какого-либо спецоборудования не предусматривалось.
Так как спецоборудование не было приобретено, данная статья не рассчитывается.

$\text{Р}_\text{с} = \text{Br } 0.00$.

\subsubsection*{Вычисление расходов на материалы и комплектующие изделия}

По статье <<Материалы и комплектующие изделия>> ($Р_\text{м}$) отражаются расходы на магнитные носители,
бумагу, красящие ленты и другие материалы,
необходимые для разработки программного продукта (ПП).
Норма расхода материалов в суммарном выражении определяется в расчете на 100 строк исходного кода по формуле (\ref{equ:Rm}).

\begin{equation}
    \label{equ:Rm}
    \text{Р}_\text{м} = \text{ЗП}_\text{осн} \cdot \frac{ \text{Н}_\text{мз} }{ 100\% } \text{,}
\end{equation}

где $\text{Н}_\text{мз}$ - норма расхода материалов от основной заработной платы, \%.

Пусть $\text{Н}_\text{мз} = 3\%$ \cite{economicNormaRashodaMaterialovVMetodichke}.

Вычисляем: $\text{Р}_\text{м} = \text{Br } 1634.64 \cdot \frac{ 3\% }{ 100\% } = \text{Br } 49.04$.

\subsubsection*{Вычисление расходов машинного времени}

Расходы по статье <<Машинное время>> ($\text{Р}_\text{мв}$) включают оплату машинного времени,
необходимого для разработки и отладки программного продукта (ПП) и вычисляется по формуле (\ref{equ:Pmv}).
Они определяются в машино-часах по нормативам на 100 строк исходного кода машинного времени
в зависимости от характера решаемых задач и типа программного продукта (ПП).

\begin{equation}
    \label{equ:Pmv}
    \text{Р}_\text{мвi} = \text{Ц}_\text{мвi} \cdot \frac{ \text{V}_\text{у} }{ 100 } \cdot \text{Н}_\text{мв} \text{, где}
\end{equation}

\begin{enumerate}
    \item[-] $\text{Ц}_\text{мвi}$ - цена одного машинного часа; 
    \item[-] $\text{V}_\text{у}$ - уточнённый общий объём строк исходного кода (LOC, lines of code);
    \item[-] $\text{Н}_\text{мв}$ - норматив расхода машинного времени на отладку 100 строк кода, машино-часов,
    где $\text{Н}_\text{мв} \in [0.6; 0.9]$.
\end{enumerate}

Текущий ноутбук имеет характеристики: $\text{V} = 19.5 \text{ В}$, $\text{I} = 2.31 \text{ А}$.

Пусть уровень расхода энергии $\text{k} = 0.8$, тогда его мощность -

$\text{P} = \text{V} \cdot \text{I} \cdot \text{k} = 19.5 \text{ В} \cdot 2.31 \text{ А} \cdot 0.8 = 36.036 \text{ Вт}$ \cite{economicCkolkoPotreblaetEnergiiNoutbyk}.

1 киловатт в час - это $\text{Br } 0.2705$ на 30 декабря 2022 \cite{economicKilovatVCasPravoBy}.

Пусть $36.036 \text{ ватт в час}$ - x.

Получаем, что $\text{Br } x = \frac{36.036 \text{ Вт } \cdot \text{ Br } 0.2705}{1000 \text{ Вт} } = \text{Br } \frac{4873869}{500000000} \approx \text{Br } 0.009748$.

Пусть 8 часов за один день, тогда за 22 дня - 176 часов ($22 \cdot 8 = 176$).

$\text{Ц}_\text{мi} = \text{Br } \frac{4873869}{500000000} \cdot 176 \approx \text{Br } 1.72$

Пусть $\text{Н}_\text{мв} = 0.6$, где $\text{Н}_\text{мв} \in [0.6; 0.9]$.

$\text{Р}_\text{мв1} = \text{Br } 1.72 \cdot \frac{ 1601 }{ 100 } \cdot 0.6 \approx \text{Br } 16.52$.

$\text{Р}_\text{мв} = \text{Р}_\text{мв1} = \text{Br } 16.52$.

\subsubsection*{Вычисление расходов на научные командировки}

Расходы по статье <<Научные командировки>> ($\text{P}_\text{нк}$)
берутся либо по смете научных командировок, разрабатываемой на предприятии,
либо в процентах от основной заработной платы исполнителей по формуле~(\ref{equ:Pnk}).
В тех случаях, когда научные командировки не предусмотрены, данная статья не расчитывается.

\begin{equation}
    \label{equ:Pnk}
    \text{Р}_\text{нк} = \text{ЗП}_\text{осн} \cdot \frac{ \text{Н}_\text{нк} }{ 100\% } \text{,}
\end{equation}

где $\text{Н}_\text{нк} \in [10; 15]$ \%.

Так как в данном проекте научные командировки не предусмотрены, данную статью не рассчитываем. 

$\text{Р}_\text{нк} = \text{Br } 0.00$.

\subsubsection*{Вычисление расходов на прочие затраты}

Расходы по статье <<Прочие затраты>> ($\text{Р}_\text{пр}$)
включают затраты на приобретение специальной научно-технической информации и специальной литературы.
Определяются по нормативу в процентах к основной заработной плате исполнителей по формле~(\ref{equ:Pnk}).

\begin{equation}
    \label{equ:Pnk}
    \text{Р}_\text{пр} = \text{ЗП}_\text{осн} \cdot \frac{ \text{Н}_\text{пр} }{ 100\% } \text{,}
\end{equation}

где $\text{Н}_\text{пр} \in [10; 15]$ \%.

Так как специальная научно-техническая информация и специальная литература не приобреталась, то данная статья не рассчитывается.

$\text{P}_\text{пз} = \text{Br }0.00$.

\subsubsection*{Вычисление расходов на накладные расходы}

Затраты по статье <<Накладные расходы>> ($\text{P}_\text{нр}$)
связаны с содержанием вспомогательных хозяйств,
а также с расходами на общехозяйственные нужды.
Определяется по нормативу в процентах к основной заработной плате исполнителей по формуле~(\ref{equ:Pnr}).
Для бюджетных организаций норматив устанавливается в пределах 100\%.

\begin{equation}
    \label{equ:Pnr}
    \text{Р}_\text{нр} = \text{ЗП}_\text{осн} \cdot  \frac{ \text{Н}_\text{нр} }{ 100\% } \text{,}
\end{equation}

где $\text{Н}_\text{нр}$ - норматив накладных расходов.

Пусть $\text{Н}_\text{нр} = 40\%$.

Вычисляем: $\text{Р}_\text{нр} = \text{Br } 1634.46 \cdot \frac{ 40\% }{ 100\% } \approx \text{Br }653.78$.

$\text{Р}_\text{нр} \approx \text{Br }653.78$.

\subsubsection*{Сумма расходов на программный продукт}

Сумма расходов служит исходной базой для расчета затрат на освоение и сопровождение программного продукта.
Она расчитываются по формуле (\ref{equ:sz}).

\begin{equation}
    \label{equ:sz}
    \text{СЗ} =
    \text{ЗП}_\text{осн}
    + \text{ЗП}_\text{доп}
    + \text{P}_\text{соц}
    + \text{P}_\text{м}
    + \text{P}_\text{с}
    + \text{P}_\text{мв}
    + \text{P}_\text{нк}
    + \text{P}_\text{пр}
    + \text{P}_\text{нр}
\end{equation}

Вычисляем: $\text{СЗ} =
\text{Br } 1634.64
+ \text{Br } 163.46
+ \text{Br } 629.32
+ \text{Br } 49.04
+ \text{Br } 0.00
\text{ } +$
$+ \text{Br } 16.32
+ \text{Br } 0.00
+ \text{Br } 0.00
+ \text{Br } 653.78
= \text{Br } 4022.53$.

$\text{СЗ} = \text{Br } 4022.53$.

\subsubsection*{Затраты на освоение программного продукта}

Затраты на освоение программного продукта ($\text{Р}_\text{о}$).
Организация\/-\hspace{0pt}разработчик участвует в освоении программного продукта (ПП) и несёт соответствующие затраты,
на которые составляется смета, оплачиваемая заказчиком по договору.
Затраты на освоение определяются по установленному нормативу от суммы затрат по формуле (\ref{equ:Po}). 

\begin{equation}
    \label{equ:Po}
    \text{Р}_\text{о} = \text{СЗ} \cdot \frac{ \text{Н}_\text{о} }{ 100\% } \text{, где}
\end{equation}

\begin{enumerate}
    \item[-] $\text{СЗ}$ - сумма расходов по статьям на разработку ПО, Br; 
    \item[-] $\text{H}_\text{o}$ - установленный норматив затрат на освоение, где $\text{H}_\text{o} \in [5; 10]\%$.
\end{enumerate}

Пусть $\text{H}_\text{o} = 5\%$.

Вычисляем: $\text{Р}_\text{о} = \text{Br } 4022.53 \cdot \frac{ 5\% }{ 100\% } \approx \text{Br } 201.13$.

\subsubsection*{Затраты на сопровождение программного продукта}

Затраты на сопровождение программного продукта ($P_{co}$).
Организация\/-\hspace{0pt}разработчик осуществляет сопровождение программного продукта (ПП) и несёт расходы,
которые оплачиваются заказчиком в соответствии с договором и сметой на сопровождение.
Для упрощения расчетов устанавливается по установленному нормативу по формуле (\ref{equ:Pco}).

\begin{equation}
    \label{equ:Pco}
    \text{Р}_\text{со} = \text{СЗ} \cdot \frac{ \text{Н}_\text{со} }{ 100\% } \text{, где}
\end{equation}

\begin{enumerate}
    \item[-] $\text{СЗ}$ - сумма расходов по статьям на разработку ПО, Br; 
    \item[-] $\text{Н}_\text{со}$ - установленный норматив затрат на сопровождение ПО, где $\text{Н}_\text{со} \in [5;10]\%$.
\end{enumerate}

Вычисляем: $\text{Р}_\text{со} = \text{Br } 4022.53 \cdot \frac{ 5\% }{ 100\% } \approx \text{Br } 201.13$.

\subsubsection*{Полная себестоимость ПП}

Полная себестоимость ($\text{СП}$) разработки программного продукта (ПП)
рассчитывается как сумма расходов по всем статьям.
Она определяется по формуле~(\ref{equ:SP}). 

\begin{equation}
    \label{equ:SP}
    \text{СП} = \text{СЗ} + \text{Р}_\text{о} + \text{Р}_\text{со}
\end{equation}

Вычисляем: $\text{СП} = \text{Br } 4022.53 + \text{Br } 201.13 + \text{Br } 201.13 = \text{Br } 4424.79$.

В результате всех расчётов полная себестоимость прогаммного продукта (ПП) составила $\text{Br } 4424.79$.

\subsection{Расчет цены и прибыли}

\subsubsection*{Рассчет плановой прибыли}

Для определения цены программного продукта необходимо расчитать плановую прибыль,
которая рассчитывается по формуле (\ref{equ:P}).

\begin{equation}
    \label{equ:P}
    \text{П} = \text{СП} \cdot \frac{ R }{ 100\% } \text{,}
\end{equation}

где $R$ - уровень рентабельности ПО, $R \in [10;30]\%$.

Пусть $R = 10\%$.

Вычисляем: $\text{П} = \text{Br } 4424.79 \cdot \frac{ 10\% }{ 100\% } \approx \text{Br } 442.48$.

\subsubsection*{Рассчет прогнозируемой цены ПП без налогов}

После расчёта прибыли от реализации определяется прогнозируемая цена программного продукта без налогов
по формуле (\ref{equ:Cep}).

\begin{equation}
    \label{equ:Cep}
    \text{Ц}_\text{п} = \text{СП} + \text{П}
\end{equation}

Вычисляем: $\text{Ц}_\text{п} = \text{Br } 4424.79 + \text{Br } 442.48 = \text{Br } 4867.27$.

\subsubsection*{Рассчет отпускной цены}

Отпускная цена (цена реализации) программного продукта (ПП) включает налог на добавленную стоимость и рассчитывается по формуле~(\ref{equ:Ceo}).

\begin{equation}
    \label{equ:Ceo}
    \text{Ц}_\text{о} = \text{СП} + \text{П} + \text{НДС}_\text{пп} \text{, где}
\end{equation}

Для данного программного продукта $\text{НДС}_\text{пп}$ рассчитывается по формуле~(\ref{equ:NDSpp}).

\begin{equation}
    \label{equ:NDSpp}
    \text{НДС}_\text{пп} = \text{Ц}_\text{п} \cdot \frac{ \text{НДС} }{ 100\% } \text{,}
\end{equation}

где $\text{НДС}$ - налог на добавленную стоимость.
В настоящее время 20\%.

% В настоящее время $\text{НДС} = 20\%$.

$\text{НДС}_\text{пп} = \text{Br } 4867.27 \cdot \frac{ 20\% }{ 100\% } = \text{Br } 973.45$;

$\text{Ц}_\text{о} = \text{Br } 4424.79 + \text{Br } 442.48 + \text{Br } 973.45 = \text{Br } 5840.72$.

\subsubsection*{Рассчёт прибыли от реализации ПО за вычетов налога на прибыль}

Прибыль от реализации программного продукта (ПП) за вычетом налога на прибыль ($\text{П}_\text{ч}$)
является чистой прибылью,
остаётся организации-разработчику
и представляет собой экономический эффект от создания нового программного продукта.
Она рассчитывается по формуле (\ref{equ:PCh}).

\begin{equation}
    \label{equ:PCh}
    \text{П}_\text{ч} = \text{П} \cdot ( 1 - \frac{ \text{Н}_\text{п} }{ 100\% }) \text{,}
\end{equation}

где $\text{Н}_\text{п}$ - ставка налога на прибыль, \%.

$\text{Н}_\text{п} = 20\%$ - ставка налога на прибыль на 17 января 2023 года \cite{economicStavkaNalogaNaPripil}.

Вычисляем: $\text{П}_\text{ч} = \text{Br } 442.48 \cdot ( 1 - \frac{ 20\% }{ 100\% }) \approx \text{Br } 353.98$.

Расчеты, связанные с ценой и прибылью ПО, представлены в таблице~\ref{tab:RaschetOtpusknoiCeniIChistoiPribiliPO}.

\begin{table}[ht]
    \centering\small

    \caption{Расчёт отпускной цены и чистой прибыли}
    \label{tab:RaschetOtpusknoiCeniIChistoiPribiliPO}

    \begin{tabular}{|l|r|l|r|}
        \hline
        \multicolumn{1}{|c|}{Наименование статей затрат}
        & \multicolumn{1}{c|}{Норматив, \%}
        & \multicolumn{1}{c|}{Расчетная формула}
        & \multicolumn{1}{c|}{Сумма затрат, Br}
        \\ \hline

        Полная себестоимость& -     & $\text{СП} = \text{СЗ} + \text{Р}_\text{о} + \text{Р}_\text{со}$                  & 4424.79   \\ \hline
        Прибыль             & 10    & $\text{П} = \text{СП} \cdot \frac{ R }{ 100\% }$                                  & 442.48    \\ \hline
        Цена без НДС        & -     & $\text{Ц}_\text{п} = \text{СП} + \text{П}$                                        & 4867.27   \\ \hline
        НДС                 & 20    & $\text{НДС}_\text{пп} = \text{Ц}_\text{п} \cdot \frac{ \text{НДС} }{ 100\% }$     & 973.45    \\ \hline
        Отпускная цена      & -     & $\text{Ц}_\text{о} = \text{СП} + \text{П} + \text{НДС}_\text{пп}$                 & 5840.72   \\ \hline
        Налог на прибыль    & 20    & $\text{П} \cdot \frac{\text{H}_\text{п} }{ 100 }$                                 & 88.50     \\ \hline
        Чистая прибыль      & 20    & $\text{П}_\text{ч} = \text{П} \cdot ( 1 - \frac{ \text{Н}_\text{п} }{ 100\% })$   & 353.98    \\ \hline
    \end{tabular}
\end{table}

В ходе произведенных расчетов определены основные экономические показатели:
полная себестоимость - Br 4424.79;
прогнозируемая цена - Br 5840.72;
чистая прибыль - Br 353.98.
При расчете цены учтены отчисления в фонд социальной защиты, а также налоги, необходимые к уплате.
К конечному итогу получаем окончательную цену продукта, равную Br 5840.72. 
