В ходе разработки данного дипломного проекта было создано мобильное приложение,
которое представляет собой альтернативу интернет-магазинам 21 век и WildBerries.
Данные магазины имеют определенный недостатк, которые повышают стоимость товаров для потребителей.

Интернет-магазин 21 век, хотя и предлагает товары по низким ценам,
требует завышения цен со стороны поставщиков,
Учитывая уже высокую стоимость турецких товаров в Беларуси,
фирма ООО "ДЕ-ПА" стремится предложить альтернативу, которая бы позволила покупателям приобретать товары,
без завышения стоимости.

Магазин WildBerries, в свою очередь, включает в цену товара стоимость его маркетинга на их площадке.
Большой ассортимент товаров, предлагаемый этим магазином, может затруднять процесс поиска и покупки товаров для пользователей.

Основная цель разработанного мобильного приложения - внедрение его в компанию и оценка эффективности его использования в дальнейшем. 

В рамках дипломного проекта была спроектирована БД и реализована серверная часть (backend)
для мобильного приложения (frontend),
которая была задокументирована в Swagger UI.

Система баз данных предоставляет удобный способ хранения информации о номенклатуре, брендах (производителях),
категориях, галерее и характеристиках номенклатуры.
Она также содержит данные о пользователях, об активациях аккаунтов, о создании сессий и об изменений электронных почт.
Созданы таблицы, предназначенные для оформления заказов, включающие документ о создании заявки и список выбранной номенклатуры.
Такая структура базы данных позволяет хранить и обрабатывать информацию эффективно и организованно.

% База данных может хранить в себе информацию о товарах в виде следующих таблиц:
% справочник производителей номенклатуры (DP\_CTL\_ItemBrands),
% справочник категорий номенклатуры (DP\_CTL\_ItemCategories),
% справочник номенклатуры (DP\_CTL\_Items),
% табличная часть с галереей номенклатуры (DP\_ LST\_ItemGalery),
% табличная часть характеристик номенклатуры (DP\_LST\_Item-Characteristics).
% и
% справочник характеристик номенклатуры (DP\_CTL\_ItemCha-racteristics).

% Для хранения данных о пользователе созданы следующие таблицы:
% справочник с даными пользователя (DP\_CTL\_Users),
% документ об активации аккаунта (DP\_DOC\_ActivationAccount),
% документ о создании сессии (DP\_DOC\_Sessi-ons),
% документ смены электронной почты (DP\_DOC\_ChangeEmail).

% Для хранения заказов созданы следующие таблицы:
% документ о заявке номенклатуры (DP\_DOC\_Orders),
% табличная часть с номенклатурой (DP\_LST\_ OrderItems).

В процессе разработки были применены токены JWT (JSON web tokens), с различными временами жизни:
токен доступа со сроком действия 3 часа;
токен обновления со сроком действия 3 дня;
токен активации аккаунта со сроком действия 24 часа;
токен для смены электронной почты со сроком действия 3 часа.
Это позволило обеспечить безопасность и защиту персональных данных пользователей при работе с приложением.
