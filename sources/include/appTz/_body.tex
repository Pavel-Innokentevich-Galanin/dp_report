\begin{ESKDtitlePage}
  \begin{flushright}
    \textbf{ПРИЛОЖЕНИЕ Г} \enspace\enspace
  \end{flushright}
  
  \begin{center}
    % \envDiplomMinistr \\
    \envDiplomEducation \\
    \envDiplomUniversity \\
    \envDiplomCathedra \\
  \end{center}

  \vfill

  \begin{center}
    \textbf{ИНСТРУКЦИЯ ПО УСТАНОВКЕ ПО}
  \end{center}

  \vfill

  \begin{center}
    \envCode \\
  \end{center}

  \vfill

  \begin{flushright}
  \begin{minipage}[t]{.49\textwidth}
    \begin{minipage}[t]{.75\textwidth}
      \begin{flushright}
        \envDiplomTeacherInfo\\
        \hspace{0pt}\\
        \envDiplomStudentInfo\\
        \hspace{0pt}\\
        Консультанты:\\
        \envDiplomEspdInfo\\
        \hspace{0pt}\\
        \envDiplomRecendentInfo\\
      \end{flushright}
    \end{minipage}
  \end{minipage}
  \begin{minipage}[t]{.49\textwidth}
    \begin{flushright}
      \begin{minipage}[t]{.75\textwidth}
        \envDiplomTeacherInitials~\envDiplomTeacherSurname\\ % Руководитель
        \hspace{0pt}\\
        \envDiplomStudentInitials~\envDiplomStudentSurname\\ % Выполнил
        \hspace{0pt}\\
        \hspace{0pt}\\ % Консультанты:
        \envDiplomEspdInitials~\envDiplomEspdSurname\\ % по ЕСПД
        \hspace{0pt}\\
        \envDiplomRecendentInitials~\envDiplomRecendentSurname\\ % рецензент
      \end{minipage}
    \end{flushright}
  \end{minipage}
\end{flushright}


  \vfill

  \begin{center}
    \ESKDtheYear
  \end{center}
\end{ESKDtitlePage}


\newpage
\tableofcontents
\hspace{0pt}\\
\newpage

\ESKDstyle{title}
\thispagestyle{plain}
\pagestyle{plain}
\hspace{0pt}

\section{Введение}

\subsection{Назначение документа}

\begin{enumerate}[label=\thesubsection.\arabic*, leftmargin=3cm]
    \item Данный документ предназначен для использования разработчиками.
    Его целью является своевременное отображение всех изменений требований к продукту,
    возникающих в процессе разработки и поддержки приложения.
    Важно, чтобы документ постоянно обновлялся,
    чтобы его содержание всегда было актуальным и информативным.
\end{enumerate}

\subsection{Описание проекта}

\begin{enumerate}[label=\thesubsection.\arabic*, leftmargin=3cm]
    \item Ознакомить покупателя с каталогом, сертификатами и прайсом.
    \item Дать пользователю информацию о кантактах фирмы.
    \item Ознакомить покупателя с номенклатурой.
    \item Клиент может добавлять номенклатуру в корзину и оформлять заявку.
\end{enumerate}

\subsection{Общие требования к проекту}

\begin{enumerate}[label=\thesubsection.\arabic*, leftmargin=3cm]
    \item Реализовать страницу номенклатуры, номенклатуры разбить по категориям, категории разбить по фирмам.
    Сделать так, чтобы можно было номенклатуре указать описание для сайта, и ключевые слова для сайта.

    \item Реализовать страницу c PDF файлами каталогов, сертификатов, прайсов.

    \item Реализовать контактов (helpers). У контактов можно задать заголовок, описание, телефоны, электронные почты, аккаунты в Skype, Viber, Whatsapp, Telegram.
    
    \item Реализовать регистрации только юридических лиц через e-mail.
    От пользователя нужны следующие данные: УНП, наименование организации, ФИО, телефон, адрес, наименование банка, расчётный счет.
     
    \item Реализовать механизм восстановления пароля.
    
    \item Реализовать механизм смены электронной почты.

    \item Локализация системы: русский.

    \item Поддержка операционной системы Android 9+.
\end{enumerate}

\section{Экраны мобильного приложения}

\subsection{Главная}

\begin{enumerate}[label=\thesubsection.\arabic*, leftmargin=3cm]
    \item Раздел доступен из нижнего меню.
    \item На главном экране выводятся списки производителей с картинкой и наименование. Можно перейти на экран с категориями производителя.
    \item На экране категорий производителя выводить картинку и наименование. Можно перейти с номенклатурой этой категории.
    \item На экране с номенклатурой показывать картинку, модель, стоимость и наименование. Можно перейти на страницу номенклатуры для ознакомления. 
\end{enumerate}

\subsection{Новости}

\begin{enumerate}[label=\thesubsection.\arabic*, leftmargin=3cm]
    \item Раздел доступен из нижнего меню.
    \item На экране новостей вывести следующие статьи: каталоги, сертификаты, прайсы, контакты.
\end{enumerate}

\subsection{Корзина}

\begin{enumerate}[label=\thesubsection.\arabic*, leftmargin=3cm]
    \item Раздел доступен из нижнего меню.
    \item Показывать номенклатуру с ценой без НДС, с НДС, и итогом (цена без НДС умноженная на количество).
    \item Реализовать возможно прибаления и убавления товара по кнопкам.
    \item Реализовать возможно ручного ввода количества.
\end{enumerate}

\subsection{Избранные}

\begin{enumerate}[label=\thesubsection.\arabic*, leftmargin=3cm]
    \item Раздел доступен из нижнего меню.
    \item Показывать номенклатуру, которую добавил пользователь в избранные.
    \item В отличии от корзины, просить пользователя авторизоваться
    (это сделано для того, чтобы не терять товары при смене устройства).
\end{enumerate}

\subsection{Аккаунт}

\begin{enumerate}[label=\thesubsection.\arabic*, leftmargin=3cm]
    \item Раздел доступен из нижнего меню.
    \item Если пользователь не авторизован, то вывести большие кнопки <<Войти в профиль>> (аутентификация) и <<Регистрация>>.
    \item Добавить ссылку на сайт, на страницу контактов.
    \item Добавить ссылку о приложении, в ктором можно проверить наличие обновлений.
    \item Экран регистрации сделать в три этапа:
    1) УНП, наименование организации, адрес, наименование банка, расчетный счёт;
    2) Фамилия, имя, отчество (при наличии), телефон;
    3) логин, e-mail, пароль.
    \item На экране аутентификации реализовать следующие ссылки:
    1) у меня уже есть аккаунт;
    2) забыли пароль.
\end{enumerate}
